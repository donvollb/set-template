$if(has-frontmatter)$
\frontmatter
$endif$
$if(title)$
$if(beamer)$
\frame{\titlepage}
$else$
\maketitle
$endif$
$if(abstract)$
\begin{abstract}
$abstract$
\end{abstract}
$endif$
$endif$



% Inhalt der Titelseite (Stichprobenbeschreibung I, hinter Titel, vor Inhaltsverzeichnis) ---------------------------

% Info.csv laden
% \DTLsetseparator{;}
% \DTLloaddb[noheader=false, keys={pdf.no, Art, pdf.name, FB.no, Campus, Campus.krz, Zeitraum, Zeitraum.krz, N.Umfragen.alle, inkl.out, Auswahl, Stichprobe.txt, FB.txt, N, Ncourses}]{info}{Info.csv}

% % Variablen definieren
% \newcommand{\Stichprobe}{\DTLfetch{info}{pdf.no}{\ParamI}{Stichprobe.txt}}
% \newcommand{\Campus}{\DTLfetch{info}{pdf.no}{\ParamI}{Campus}}
% \newcommand{\Zeitraum}{\DTLfetch{info}{pdf.no}{\ParamI}{Zeitraum}}
% \newcommand{\NUmfragenAlle}{\DTLfetch{info}{pdf.no}{\ParamI}{N.Umfragen.alle}}
% \newcommand{\Ncourses}{\DTLfetch{info}{pdf.no}{\ParamI}{Ncourses}}
% \newcommand{\Nvotes}{\DTLfetch{info}{pdf.no}{\ParamI}{N}}

% % Ausgabe der Stichprobenbeschreibung I
% \textbf{\Large{Stichprobenbeschreibung I}}

% \vspace{2mm}

% Fachbereich: \tabto{3cm} \textit{\Stichprobe}

% Campus: \tabto{3cm} \textit{\Campus}

% Zeitraum: \tabto{3cm} \textit{\Zeitraum}

% Fragebogen: \tabto{3cm} \textit{LVE Basis}

% \vspace{12mm}

% \textbf{Anzahl insgesamt erstellter LVE-Umfragen o.g. Zeitraums: } N\textsubscript{surveys} = \NUmfragenAlle

% \textbf{Anzahl LVE-Umfragen, die in diesen Bericht eingingen (Mindestrücklauf 3 Stimmen): } N\textsubscript{courses} = \Ncourses

% \textbf{Anzahl Stimmen, die in diesen Bericht eingingen: } N\textsubscript{votes} = \Nvotes

% \pagebreak

% -----------------------------------------------------------------------------


