% Pakete
\usepackage{fancyhdr}             % für Kopf- und Fußzeile
\usepackage{graphicx}             % für das Einbinden von Graphiken
\usepackage{eurosym}              % für Eurosymbole
\usepackage{booktabs}             % für zusätzliche Stilelemente in Tabellen
\usepackage{xcolor}               % für Farben
\usepackage{tabu}                 % für Tabellen, die über mehrere Seiten gehen
\usepackage{tabto}                % für Tabs
\usepackage{makecell}             % für Layouts in Tabellen
\usepackage{longtable}            % für Tabellen über mehrere Seiten
\usepackage{array}                % ebenfalls Tabellenlayout/Matritzen
\usepackage{multirow}             % für Tabellen mit mehrzeiligen Zellen
\usepackage{wrapfig}              % für Bilder und Tabellen, die von text umschlossen werden
\usepackage{float}                % für Abbildungen/Tabellen
\usepackage{colortbl}             % für farbige Zeilen/Spalten in Tabellen
\usepackage{pdflscape}            % für mögliche Rotation der Seiten
\usepackage{threeparttable}       % für Titel/Anmerkungen von Tabellen
\usepackage[normalem]{ulem}       % für Unterstriche
\usepackage{fontspec}             % für Schriftarten
\usepackage{hyperref}             % für Hyperlinks
\usepackage{geometry}             % für Seitenränder
\usepackage{tocloft}              % für Inhaltsverzeichnismodifikationen
\usepackage{titlesec}             % für Überschriftenmodifikationen
\usepackage{datatool}             % für Laden der Info.csv im Partial Template (before-body.tex)

 % Ab hier beginnt das Setup für die Hyperlinks
\hypersetup{
colorlinks=true,                  % farbige Links?
linkcolor=blue,                   % Farbe der Hyperlinks zu Abschnitten innerhalb des Dokuments
filecolor=magenta,                % Farbe von Hyperlinks zu Dateien
urlcolor=cyan}                    % Farbe von URLs



  % Hier wird noch das Inhaltsverzeichnis ein wenig zurechtgestutzt
\renewcommand{\cftsecleader}{\cftdotfill{\cftdotsep}}
% Punkte zwischen Eintrag und Seite auch bei "Sections" (erste Überschriftsebene)
\renewcommand\cftsecfont{\mdseries}
% Einträge bei "Sections" werden nicht fett geschrieben
\renewcommand\cftsecpagefont{\mdseries}
% Seitenzahlen bei "Sections" werden nicht fett geschrieben
\setlength{\cftsecindent}{1.5em}                      % Einträge im Inhaltsverzeichnis leicht eingerückt


% Hier wird das Layout der Section-Überschriften verändert (zentriert)
\titleformat{\section}
{\normalfont\Large\bfseries\center}
{\thesection}{1em}{}
  
  
% Hier werden einige Farben für die Verwendung im LATEX-Code definiert

% blau-grau (Schiefer)
\definecolor{bluegray}{HTML}{507289}             
\definecolor{bluegraytable}{HTML}{EFF2F4}       

% grün-grau (Ozean)
\definecolor{greengray}{HTML}{77b6ba}             
\definecolor{greengraytable}{HTML}{F3F8F9}       

% dunkelblau (Nacht)
\definecolor{darkblue}{HTML}{042c58}             
\definecolor{darkbluetable}{HTML}{E8ECF0} 

% hellblau (Tag)
\definecolor{lightblue}{HTML}{6ab2e7}             
\definecolor{lightbluetable}{HTML}{F2F8FD} 

% dunkelgrün (Petrol)
\definecolor{darkgreen}{HTML}{006b6b}             
\definecolor{darkgreentable}{HTML}{9AA8AE} 

% hellgrün (Apfel)
\definecolor{lightgreen}{HTML}{26d07c}             
\definecolor{lightgreentable}{HTML}{EBFBF3} 

% violett (Pflaume)
\definecolor{violett}{HTML}{4c3575}             
\definecolor{violetttable}{HTML}{EFEDF3} 

% pink (Fuchsia)
\definecolor{pink}{HTML}{d13896}             
\definecolor{pinktable}{HTML}{FBEDF6} 

% rot (Himbeere)
\definecolor{red}{HTML}{e31b4c}             
\definecolor{redtable}{HTML}{FCEAEF} 

% orange (Mango)
\definecolor{orange}{HTML}{ffa252}             
\definecolor{orangetable}{HTML}{FFF7EF} 

% Allgemeine Farbe
\definecolor{darkgrey}{RGB}{200, 200, 200}        % Dunkelgrau für manche Beschriftungen
\definecolor{steelblue}{RGB}{70,130,180}          % Schriftfarbe steelblue

% Standardfarben, falls nicht anders definiert
\colorlet{maincolor}{bluegray}
\colorlet{tablecolor}{bluegraytable}       

% Falls eine spezifische Farbe (der definierten) genannt wird, werden die Standardfarben überschrieben
$if(color)$
\colorlet{maincolor}{$color$}
\colorlet{tablecolor}{$color$table}  
$endif$




% Pagestyle, Kopf- und Fusszeile ----------------------------------------

% fancy als pagestyle
\pagestyle{fancy}
\fancyhf{}

$if(lheadlogo)$
$if(lheadheight)$
\lhead{\includegraphics[height=$lheadheight$]{$lheadlogo$}}
$else$
\lhead{\includegraphics[height=0.7cm]{$lheadlogo$}}
$endif$
$endif$

$if(rheadlogo)$
$if(rheadheight)$
\rhead{\includegraphics[height=$rheadheight$]{$rheadlogo$}}
$else$
\rhead{\includegraphics[height=1.1cm]{$rheadlogo$}}
$endif$
$endif$

% Horizontale Trennstriche
\renewcommand{\headrulewidth}{1pt}
\renewcommand{\headrule}{\hbox to\headwidth{%
    \color{maincolor}\leaders\hrule height \headrulewidth\hfill}}
    
\renewcommand{\footrulewidth}{1pt}
\renewcommand{\footrule}{\hbox to\headwidth{%
    \color{maincolor}\leaders\hrule height \footrulewidth\hfill}}
    
% Fußzeile
\rfoot{\color{darkgray}{Seite \thepage}}

$if(lfoot)$
\lfoot{\color{darkgray}{$lfoot$}}
$endif$


% Pagestyle plain umdefinieren fuer Titelseite und Inhaltsverzeichnis (Seite 1 und 2) --------------------------
\fancypagestyle{plain}{
  \fancyhf{}
  
$if(lheadlogo)$
$if(lheadheight)$
\lhead{\includegraphics[height=$lheadheight$]{$lheadlogo$}}
$else$
\lhead{\includegraphics[height=0.7cm]{$lheadlogo$}}
$endif$
$endif$

$if(rheadlogo)$
$if(rheadheight)$
\rhead{\includegraphics[height=$rheadheight$]{$rheadlogo$}}
$else$
\rhead{\includegraphics[height=1.1cm]{$rheadlogo$}}
$endif$
$endif$

% Horizontale Trennstriche
\renewcommand{\headrulewidth}{1pt}
\renewcommand{\headrule}{\hbox to\headwidth{%
    \color{maincolor}\leaders\hrule height \headrulewidth\hfill}}
    
\renewcommand{\footrulewidth}{1pt}
\renewcommand{\footrule}{\hbox to\headwidth{%
    \color{maincolor}\leaders\hrule height \footrulewidth\hfill}}
    
% Fußzeile
\rfoot{\color{darkgray}{Seite \thepage}}

$if(lfoot)$
\lfoot{\color{darkgray}{$lfoot$}}
$endif$

}